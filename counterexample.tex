\documentclass[oneside,draft]{amsart}
\usepackage{amsmath,amssymb,amsthm,stmaryrd}
\usepackage[final]{hyperref}
\usepackage[final]{graphicx}
\usepackage{xcolor}
\usepackage{fixme}
\fxsetup{draft}
\fxusetheme{color}

\newcommand{\setcond}[2]{\left\{#1\mid#2\right\}}
\newcommand{\norm}[1]{\lVert #1\rVert}
\newcommand{\diam}{\operatorname{diam}}

\newtheorem{thm}{Theorem}
\newtheorem{lem}{Lemma}
\newtheorem{prop}{Proposition}
\newtheorem{note}{Note}
\newtheorem{obvious}{Obvious}
\newtheorem{cor}{Corollary}
\newtheorem{claim}{Claim}
\newtheorem{rem}{Remark}
\newtheorem{defn}{Definition}
\newtheorem{exer}{Exercise}

\title{A counterexample against Kakeya's conjecture}
\author{Victor Porton, ORCID 0000-0001-7064-7975}
\email{porton.victor@gmail.com}
% \subjclass[2020]{28D99, 51M05, 54J99, 57N99, 51M15, 51M99}
\keywords{Kakeya conjecture, dimensionality, geometry, general topology, long-standing open problem}

\begin{document}

\begin{abstract}
I give a counterexample against Kakeya's conjecture for~$n=2$, signifying that the classic proof was in error.
\end{abstract}

\maketitle  

\section{Intro}

It can be seen from the drawing~\ref{fig:example}, that Kakeya conjecture is false for~$n=2$. Below I will present a rigorous proof of this.

Therefore the classic proof~\cite{kakeya2d} is wrong. Many articles need to be retracted.

\begin{figure}[hbt]
    \centering
    \includegraphics[width=\textwidth]{cantor2}
    \caption{Counterexample to Kakeya for~$n=2$. The blue is the Cantor set, the red is a Besicovich set}
    \label{fig:example}
\end{figure}

\section{The problem}

I will point a counterexample against Kakeya's conjecture for~$n=2$.

\textbf{Kakeya's conjecture for $n=2$.} A union of unit-length intervals on a plane has Hausdorff dimension~$2$, if there are intervals of all possible directions among them.

The counterexample:

Let $x = \sum_{i\in\mathbb{N}} \frac{x_i}{2^i} \mapsto \sum_{i\in\mathbb{N}} \frac{2\cdot x_i}{3^i} = c(x)$ be the usual mapping from reals to the Cantor set.
The counterexample is the union~$U=\bigcup_{x\in[0;1[}V_x$ of intervals \[ V_x = \setcond{(c(x)-t\cos(\pi x), t\sin(\pi x))}{t\in\left]0;1\right[}. \]

\section{The proof}

First, it is easy to see that segments~$V_x$ don't intersect each other.

Let vertical intervals \[ V'_x = \setcond{(c(x), t)}{t\in\left]0;1\right[}. \]

Using~\cite{189275} we easily deduce that $W=\bigcup_{x\in[0;1[}V'_x$ has dimension~$1+\alpha$ where $\alpha=\frac{\ln 2}{\ln 3}$ is the dimensionality of the Cantor set. We will prove, that $U$~and~$W$ are diffeomorphic to finish the proof. The diffeomorphism is given by the mapping \[ (c,t) \mapsto (c-t\cos(\pi x), t\sin(\pi x)). \]

It is an obvious bijection. That it is~$C^1$-smooth i both directions in is also obvious.

\section{``Political'' context}

I discovered ordered semigroup/semicategory actions in 2019. The world didn't react obeying like a sheep to decision of Russian Orthodoxes to kick people like me from normal life such as academic carrier. That's scary: the most valuable component is a missing component. Missing ordered semigroup actions amount to like half of world economy missing~\cite{osa-important}.

I don't know how Kakeya conjecture is used in the rest of math, but I use it to build a bridge between two about halves: academic math and the second lost half, through my glorification.

The advice: ``Solve some famous open problem to prove that you are a good mathematician.'' sounded for me like a mocking. But in the new reality, I indeed did it.

\bibliographystyle{amsplain}
\bibliography{kakeya}

\end{document}
