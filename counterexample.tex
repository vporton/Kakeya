\documentclass[oneside,draft]{amsart}
\usepackage{amsmath,amssymb,amsthm,stmaryrd}
\usepackage[final]{hyperref}
\usepackage[final]{graphicx}
\usepackage{xcolor}
\usepackage{fixme}
\fxsetup{draft}
\fxusetheme{color}

\newcommand{\setcond}[2]{\left\{#1\mid#2\right\}}
\newcommand{\norm}[1]{\lVert #1\rVert}
\newcommand{\diam}{\operatorname{diam}}

\newtheorem{thm}{Theorem}
\newtheorem{lem}{Lemma}
\newtheorem{prop}{Proposition}
\newtheorem{note}{Note}
\newtheorem{obvious}{Obvious}
\newtheorem{cor}{Corollary}
\newtheorem{claim}{Claim}
\newtheorem{rem}{Remark}
\newtheorem{defn}{Definition}
\newtheorem{exer}{Exercise}

\title{A counterexample against Kakeya's conjecture}
\author{Victor Porton, ORCID 0000-0001-7064-7975}
\email{porton.victor@gmail.com}
\subjclass[2020]{51N20, 54H99, 28A78}
\keywords{Kakeya conjecture, dimensionality, geometry, general topology, long-standing open problem}

\begin{document}

\begin{abstract}
I give a counterexample against Kakeya's conjecture for~$n=2$, signifying that the classic proof was in error.
\end{abstract}

\maketitle  

\section{Intro}

It can be seen from the drawing~\ref{fig:example}, that Kakeya conjecture is false for~$n=2$. Below I will present a rigorous proof of this.

\begin{figure}[hbt]
    \centering
    \includegraphics[width=\textwidth]{cantor2}
    \caption{Counterexample to Kakeya for~$n=2$.\\The blue is the Cantor set, the red is a Besicovich set}
    \label{fig:example}
\end{figure}

Therefore the classic proof~\cite{kakeya2d} is wrong. Many articles need to be retracted.

\section{The problem}

I will point a counterexample against Kakeya's conjecture for~$n=2$.

\textbf{Kakeya's conjecture for $n=2$.} A union of unit-length intervals on a plane has Hausdorff dimension~$2$, if there are intervals of all possible directions among them.

The counterexample:

Let \[ x = \sum_{i\in\mathbb{N}} \frac{x_i}{2^i} \mapsto \sum_{i\in\mathbb{N}} \frac{2\cdot x_i}{3^i} = c(x) \] be the usual mapping from reals to the Cantor set.
The counterexample is the union~$U=\bigcup_{x\in[0;1[}S_x$ of intervals \[ S_x = \setcond{(c(x)-t\cos(\pi x), t\sin(\pi x))}{t\in\left]0;1\right[}. \]

\section{The proof}

Obviously, $U$~contains intervals of all directions.

First, it is easy to see that segments~$S_x$ don't intersect each other.

Partition $\left[0;1\right[$ into~$2^m$ intervals $I_s=\left[\frac{k}{2^m};\frac{k+1}{2^m}\right[$, $k=0,\ldots,2^m-1$, each of length $1/2^m$, based on the first~$m$ binary digits $(s_1,\ldots,s_m)$. Define $U_s=\bigcup_{x\in I_s} S_x$, so $U=\bigcup_s U_s$.

For $x\in I_x$, $c(x)=a_s+r$, where $a_s=\sum_{i=1}^m \frac{2s_i}{3^i}$ and $r=\sum_{i=m+1}^\infty \frac{2x_i}{3^i}\leq 1/3^m$, so $c(x)\in[a_s;a_s+1/3^m]$. Approximate~$U_s$ with a paralellogram~$T_s$:
\[
T_s = \bigcup_{c\in[a_s;a_s+1/3^m]} \setcond{(c-t\cos(\pi x_s), c+t\sin(\pi x_s)}{t\in[0;1]},
\]
where $x_s=k/2^m$. This paralellogram has base $1/3^m$ and height $\lvert\sin(\pi x_s)\rvert$ with area $O(1/3^m)$.

For $x\in I_s$, $\lvert\cos(\pi x)-\cos(\pi x_s)\rvert$ and $\lvert\sin(\pi x)-\sin(\pi x_s)\rvert$ are (mean value theorem) $O(1/2^m)$, so $U_s$ lies within a $\delta$-neigh\-bor\-hood of~$T_s$, with $\delta=O(1/2^m)$.

Cover each~$T_s$ with balls of diameter $\epsilon=1/3^m$. The number of balls is $O(\mathrm{area}/\epsilon^2)$. Including the neighborhood, adjust the cover for~$U_s$, but the key is the asymptotic cost:

\begin{itemize}
\item Number of~$U_s$: $2^m$,
\item Balls per~$U_s$: $O(3^m)$,
\item Sum of~$(\diam)^2$: $2^m\cdot O(3^m)\cdot(1/3^m)^2=O((2/3)^m)$.
\end{itemize}

As $m\to\infty$, $(2/3)^m\to 0$ and $\epsilon\to 0$, so the Hausdorff measure $H^2(U)=0$.

So $\dim U<2$.

\section{``Political'' context}

I discovered ordered semigroup/semicategory actions in 2019. The world didn't react obeying like a sheep to decision of Russian Orthodoxes to kick people like me from normal life such as academic carrier. That's scary: the most valuable component is a missing component. Missing ordered semigroup actions amount to like half of world economy missing~\cite{osa-important}.

I don't know how Kakeya conjecture is used in the rest of math, but I use it to build a bridge between two about halves: academic math and the second lost half, through my glorification.

The advice: ``Solve some famous open problem to prove that you are a good mathematician.'' sounded for me like a mocking. But in the new reality, I indeed did it.

\bibliographystyle{amsplain}
\bibliography{kakeya}

\end{document}
