\documentclass{amsart}
\usepackage{amsmath,amssymb,amsthm}
\usepackage{hyperref}

\newcommand{\funcoids}{\mathsf{FCD}}
\newcommand{\supfun}[1]{\left\langle#1\right\rangle}

\newtheorem{theorem}{Theorem}

\title{Proof of Kakeya's conjecture}
\author{Victor Porton, ORCID 0000-0001-7064-7975}
\email{porton.victor@gmail.com}

\begin{document}

\maketitle  

\emph{This is a preliminary draft sketch of my proposed proof of Kakeya conjecture. Report any errors, please.}

The proof uses theory of funcoids~\cite{volume-1}. It looks like that funcoids can be easily excluded from the proof, making the proof elementary. In this case, the role of funcoids is to provide intuition that helped to find the proof.

\textbf{Kakeya set conjecture:} Define a \emph{Besicovitch} set in $\mathbb{R}^n$ to be a set which contains a unit line segment in every direction. Is it true that such sets necessarily have Hausdorff dimension and Minkowski dimension equal to~$n$?

\section{Kakeya proof (rough draft)}

\subsection{Spherical coordinates}

I will denote $n$-dimensional spherical coordinates as~$\theta=(r, \theta_1,\dots,\theta_{n-1})$.
I also denote direction $\theta'=(\theta_1,\dots,\theta_{n-1})$.

Here $r\geq 0$; $\theta_i\in[0;\pi]$ for $i=1,\dots,n-2$; $\theta_{n-1}\in[0;2\pi\mathclose[$ .

Mapping $\Sigma_n$ from spherical coordinates to Euclidean coordinates is defined by the formulas:
\[
\begin{aligned}x_{1}&=r\cos(\theta _{1}),\\x_{2}&=r\sin(\theta _{1})\cos(\theta _{2}),\\x_{3}&=r\sin(\theta _{1})\sin(\theta _{2})\cos(\theta _{3}),\\&\qquad \vdots \\x_{n-1}&=r\sin(\theta _{1})\cdots \sin(\theta _{n-2})\cos(\theta _{n-1}),\\x_{n}&=r\sin(\theta _{1})\cdots \sin(\theta _{n-2})\sin(\theta _{n-1}).\end{aligned}
\]

Accordingly~\cite{polar-jacobian} the determinant of Jacobian of this transformation is
\[ \det (J\Sigma_n)(\theta) = r\sin\theta_1\dots\sin\theta_{n-2}\det (J\Sigma_{n-1})(\theta). \]
Therefore
\[
\det (J\Sigma_n)(\theta)\ne 0\Leftrightarrow 0\notin\{r,\sin\theta_1,\dots,\sin\theta_{n-2}\}.
\]
So, $\det (J\Sigma_n)(\theta)\ne 0$ almost everywhere.

Mapping $\Pi_n$ from Euclidean coordinates to spherical coordinates is defined by the formulas:
\[
\begin{aligned}r&={\textstyle {\sqrt {{x_{n}}^{2}+{x_{n-1}}^{2}+\cdots +{x_{2}}^{2}+{x_{1}}^{2}}}},\\\theta _{1}&=\operatorname {atan2} \left({\textstyle {\sqrt {{x_{n}}^{2}+{x_{n-1}}^{2}+\cdots +{x_{2}}^{2}}}},x_{1}\right),\\\theta _{2}&=\operatorname {atan2} \left({\textstyle {\sqrt {{x_{n}}^{2}+{x_{n-1}}^{2}+\cdots +{x_{3}}^{2}}}},x_{2}\right),\\&\qquad \vdots \\\theta _{n-2}&=\operatorname {atan2} \left({\textstyle {\sqrt {{x_{n}}^{2}+{x_{n-1}}^{2}}}},x_{n-2}\right),\\\theta _{n-1}&=\operatorname {atan2} \left(x_{n},x_{n-1}\right).\end{aligned}
\]  

[TODO: Jacobian of $\Pi_n$.]

\subsection{The proof}

Let $K$ be a Besicovitch set in $\mathbb{R}^n$.

I will call the \emph{right vicinity} or \emph{one-si\-de vicinity} of a point~$a$ the filter~$\supfun{\Delta_{\geq}}\{a\}$.

We will do our geometry in $\mathbb{R}^{n+1}$.

It's enought to prove that join of one-si\-de vicinities on $S^n$ has dimension~$n$, provided that vicinities are of all directions
of an arbitrarily chosen hyperplane~$\mathbb{R}^n$.
(Here dimension of a funcoid is defined as least dimension of a set above it.)

Without loss of generality, we can assume that all hairs point outside of $S^n$, because we can replace any vicinities to ``opposite direction'' vicinity without influencing trueness of the original Kakeya statement. Then hairs are preserved [they are deformed! need to take this into account] (up to a translation in space) by this.

Let map the sphere together with its hairs to coordinates center by the formula $r\mapsto r-1$ for $r\geq 1$ (leaving $\theta'$ the same).
This preserves the dimension because the Jacobian matrix has full rank almost everywhere.

Thus we have enough to prove that join of one-si\-de vicinities of the same point (the center of coordinates) is~$n$.

To prove this use di\-men\-si\-on-pre\-ser\-ving map from spherical corrdinates to Euclidean coordinates.

We've got the funcoidal product (e.g.\ subatomic product) of $n$\ vicinities what is obviously of dimension~$n$.

\bibliographystyle{amsplain}
\bibliography{kakeya}

\end{document}