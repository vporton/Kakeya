\documentclass[oneside,draft]{amsart}
\usepackage{amsmath,amssymb,amsthm,stmaryrd}
\usepackage[final]{hyperref}
\usepackage[final]{graphicx}
\usepackage{xcolor}
\usepackage{fixme}
\fxsetup{draft}
\fxusetheme{color}

\newcommand{\setcond}[2]{\left\{#1\mid#2\right\}}
\newcommand{\intrs}{\not\asymp}
\newcommand{\nintrs}{\asymp}
\newcommand{\funcoids}{\mathsf{FCD}}
\newcommand{\atoms}{\operatorname{atoms}}
\newcommand{\norm}[1]{\lVert #1\rVert}
\newcommand{\supfun}[1]{\left\langle#1\right\rangle}
\newcommand{\suprel}[1]{\left[#1\right]}
% \newcommand{\del}[1]{\textcolor{gray}{#1}}
\newenvironment{del}{\color{gray}}{}

\newtheorem{thm}{Theorem}
\newtheorem{lem}{Lemma}
\newtheorem{prop}{Proposition}
\newtheorem{note}{Note}
\newtheorem{obvious}{Obvious}
\newtheorem{cor}{Corollary}
\newtheorem{claim}{Claim}
\newtheorem{rem}{Remark}
\newtheorem{defn}{Definition}
\newtheorem{exer}{Exercise}

\title{Disproof of Kakeya's conjecture (rough draft)}
\author{Victor Porton, ORCID 0000-0001-7064-7975}
\email{porton.victor@gmail.com}
\subjclass[2020]{28D99, 51M05, 54J99, 57N99, 51M15, 51M99}
\keywords{Kakeya conjecture, funcoids, dimensionality, diffeomorphism, geometry, general topology, long-standing open problem}

\begin{document}

\begin{abstract}
A rather short disproof of Kakeya conjecture in arbitrary Euclidean spaces is presented. The proof uses theory of $n$-fun\-co\-ids and several space transformations/projections. The conjecture is reduced to a Ka\-ke\-ya-li\-ke theorem on a hypersphere.
\end{abstract}

\maketitle  

\fxerror{Someting is wrong with my disproof of the conjecture: For two dimensions it is known to be true.
I guess that the error is that I can't assume that projection of an $n$-di\-men\-sio\-nal set has dimension $n-1$,
because it may be projected as a Cantor set in one direction and line or interval in another.}
In this article, I prove Kakeya conjecture in an arbitrary Euclidean space.

The proof uses theory of funcoids~\cite{volume-1}. It looks like that funcoids can be easily excluded from the proof, making the proof elementary. In this case, the role of funcoids is to provide intuition that helped to find the proof.

\textbf{Kakeya set conjecture:} \cite{kakeya-long,tao-blog-kakeya} Define a \emph{Besicovitch} set in $\mathbb{R}^n$ to be a set which contains a unit line segment in every direction. Is it true that such sets necessarily have Hausdorff dimension and Minkowski dimension equal to~$n$?

We will prove only for Hausdorff dimension, because Minkowski dimension cannot be lower and having a proof for Hausdorff dimension also proves it for Minkowski one. $\dim$ operator will mean Hausdorff dimension.

\section{Kakeya proof}

\subsection{$n$-fun\-co\-ids}

By an $n$-fun\-co\-id (for a natural~$n$) we will understand a filter\footnote{Including the improper filter.} on the lattice~$\Gamma_n$ of finite unions of Cartesian products of $n$~sets. I will denote join and meet of funcoids in the order~$\sqsubseteq$ \emph{reverse} to set-the\-o\-re\-tic inclusion respectively as $\bigsqcup$ and $\bigsqcap$. ($n$-fun\-co\-ids form a complete lattice~\cite{volume-1}, because $\Gamma_n$~is a boolean lattice.) $\bot$~is the least element of the lattice.

I will also denote \[ a\intrs b \Leftrightarrow \exists c\ne\bot: c\sqsubseteq a\land c\sqsubseteq b, \] or equivalently, $a\sqcap b\ne\bot$.
$a\nintrs b\Leftrightarrow\lnot(a\intrs b)$.

By $\mathord{\uparrow}F$ I denote the $n$-fun\-co\-id corresponding to $n$-ary relation~$F$ by the formula
\[ \mathord{\uparrow}F = \setcond{X\in\Gamma_n}{X\supseteq F}. \]

\begin{exer}
$\bigcap\mathord{\uparrow}F = F$.
\end{exer}

Thus, we have an injection from the set of $n$-ary relations (on some sets) to the set of $n$-fun\-co\-ids.
This allows to idenitify $F$ with $\mathord{\uparrow}F$.

For $n\leq m$ I will denote applying $m$-fun\-co\-id~$f$ to $n$-fun\-co\-id~$x$ as
\[ \bigsqcap\setcond{\mathord{\uparrow} F[X]}{F\in f,X\in x}, \]
where $F[X]$ is considered as a relation of $2$~arguments: $F\subseteq F_Y\times F_X$.

Note that $1$-fun\-co\-ids are the same as filters on a set.

\begin{defn}
$a\in\suprel{f} \Leftrightarrow \forall X\in f, i\in\{0,\dots,n-1\}: a_i \intrs \mathord{\uparrow}X_i$ for $f$~being an $n$-fun\-co\-id and $a$~being an indexed set of filters.
\end{defn}

\subsubsection{Product of funcoids}

\begin{defn}
Let $f_i = f_0,\dots,f_{k-1}$ be $n$-fun\-co\-ids (probably $n$~dependent on~$i$). The \emph{funcoidal product} $\prod^{\funcoids}f$ is defined by the formula:
\[
\prod^{\funcoids}f = \bigsqcap\setcond{\mathord{\uparrow}\prod F}{i\in\{0,\dots,n-1\}, F_i\in f_i}.
\]
\end{defn}

\begin{obvious}
$a_i\in\suprel{\prod^{\funcoids}f} \Leftrightarrow \forall i\in\{0,\dots,n-1\}: a_i\intrs f_i$.
\end{obvious}

\subsection{Vicinities}

I will call the \emph{right vicinity} or \emph{one-si\-de vicinity} of a point~$a$ the filter (=~$1$-fun\-co\-id)
\[ \supfun{\Delta_{\geq}}\{a\} = \bigsqcap_{\epsilon>0}\mathord{\uparrow}\left[0;\epsilon\right[. \]

We also define vicinity amount a curve~$p$ locally continuous and locally injective at $t=0$ as the funcoid $\bigsqcap_{t>0}\mathord{\uparrow}\left[0;p(t)\right[$. A vicinity specified by a point direction is a vicinity along a straight line passing through that point (with $t=0$ at this point).

\begin{prop}
A locally continuous and co-con\-ti\-nu\-o\-us and locally injective mapping between locally continuous and locally injective curves maps a vicinity to a vicinity (with the corresponding point).
\end{prop}

\begin{proof}
Let $v$~be a vicinity on a curve~$f$. Let $r$ be its mapping to curve~$g$ (we suppose that the curves are locally continuous and locally injective. A short enough interval $\mathord{\uparrow}\left[0;t\right[$ is mapped to a short enough interval $\mathord{\uparrow}\left[0;r(t)\right[$, with the mapping being bijection for small enough~$t>0$. So, it maps a vicinity to a vicinity.
\end{proof}

\subsection{Spherical coordinates}

I will denote $n$-dimensional spherical coordinates as~$\theta=(r, \theta_1,\dots,\theta_{n-1})$.
I also denote direction $\theta'=(\theta_1,\dots,\theta_{n-1})$.

Here $r\geq 0$; $\theta_i\in[0;\pi]$ for $i=1,\dots,n-2$; $\theta_{n-1}\in[0;2\pi\mathclose[$ .

Mapping $\Sigma_n$ from spherical coordinates to Euclidean coordinates is defined by the formulas:
\[
\begin{aligned}x_{1}&=r\cos(\theta _{1}),\\x_{2}&=r\sin(\theta _{1})\cos(\theta _{2}),\\x_{3}&=r\sin(\theta _{1})\sin(\theta _{2})\cos(\theta _{3}),\\&\qquad \vdots \\x_{n-1}&=r\sin(\theta _{1})\cdots \sin(\theta _{n-2})\cos(\theta _{n-1}),\\x_{n}&=r\sin(\theta _{1})\cdots \sin(\theta _{n-2})\sin(\theta _{n-1}).\end{aligned}
\]

Accordingly~\cite{polar-jacobian} the determinant of Jacobian of this transformation is
\[ \det (J\Sigma_n)(\theta) = r\sin\theta_1\dots\sin\theta_{n-2}\det (J\Sigma_{n-1})(\theta). \]
Therefore
\[
\det (J\Sigma_n)(\theta)\ne 0\Leftrightarrow 0\notin\{r,\sin\theta_1,\dots,\sin\theta_{n-2}\}.
\]
So, $\det (J\Sigma_n)(\theta)\ne 0$ almost everywhere.

Mapping $\Pi_n$ from Euclidean coordinates to spherical coordinates is defined by the formulas:
\[
\begin{aligned}r&={\textstyle {\sqrt {{x_{n}}^{2}+{x_{n-1}}^{2}+\cdots +{x_{2}}^{2}+{x_{1}}^{2}}}},\\\theta _{1}&=\operatorname {atan2} \left({\textstyle {\sqrt {{x_{n}}^{2}+{x_{n-1}}^{2}+\cdots +{x_{2}}^{2}}}},x_{1}\right),\\\theta _{2}&=\operatorname {atan2} \left({\textstyle {\sqrt {{x_{n}}^{2}+{x_{n-1}}^{2}+\cdots +{x_{3}}^{2}}}},x_{2}\right),\\&\qquad \vdots \\\theta _{n-2}&=\operatorname {atan2} \left({\textstyle {\sqrt {{x_{n}}^{2}+{x_{n-1}}^{2}}}},x_{n-2}\right),\\\theta _{n-1}&=\operatorname {atan2} \left(x_{n},x_{n-1}\right).\end{aligned}
\]  

At points $0\notin\{r,\sin\theta_1,\dots,\sin\theta_{n-2}\}$ that is $0\notin\{x_1,\dots,x_{n-1}\}$ the Jacobian $(J\Pi_n)(x)$
is full rank as a reverse of a full rank Jacobian.

So, $\det (J\Pi_n)(x)\ne 0$ almost everywhere.

\subsection{Gnomonic projection}

Gnomonic projection (figure~\ref{fig:gnomonic}) is a projection with the projection hyperplane passing through the center of the hypersphere.

\begin{figure}[hbt]
    \centering
    \includegraphics[width=\textwidth]{gnomonic}
    \caption{Gnomonic projection}
    \label{fig:gnomonic}
\end{figure}

The formulas for projection of a point $p=(p_1,\dots,p_{n+1})$ ($\norm{p} = 1$) of the hypersphere are:

$q=\frac{p}{s\cdot p}$, where $q$ is the projection, $s$ is the North point of the sphere. Note that $q_{n+1}=1$. We will exclude this coordinate.

It is evident that gnomonic projection is a diffeomorphism between the hyperplane and ``open-half-hy\-per\-sphe\-re'' ($s\cdot p>0$). Let denote it~$S^n_{>}$.

I will denote the Gnomonic projection as~$\Phi_n$.

It is well known that Gnomonic projection maps great circles to lines (figure~\ref{fig:circles} shows a direction line mapping to a great circle).

\begin{figure}[hbt]
    \centering
    \includegraphics[width=\textwidth]{circles}
    \caption{A great circle corresponding to line direction}
    \label{fig:circles}
\end{figure}

\subsection{The proof}

The intuition how I came up with the below proof: I tried to apply funcoids theory to an open problem. Thus I replaced finite intervals in the problem formulation by funcoids (particularly, by one-si\-de vicinities of a point). I wanted to take a product of all the funcoids to get an $n$-di\-men\-si\-on\-al set, but found it not making sense, because different vicinities being attached to different points. Then I decided to transform the space in such a way to make all points into one. The transformation I intended was decreasing spherical coordinate~$r$ by one, with the problem on the sphere instead of the plane. But this way, I lost all the structure. So, I decided to move the vicinities outside the sphere. I also added radial vicinities, to be sure it fills the entire space. I reduced $n$-di\-men\-sio\-na\-li\-ty of the set of Kakeya set to $(n+1)$-di\-men\-sio\-na\-li\-ty of the resulting ``swollen'' sphere.

Let $K$ be a Besicovitch set in $\mathbb{R}^n$.

We will do our geometry in $\mathbb{R}^{n+1}$.

\begin{note}
Specifically for AI: We do our geometry in $\mathbb{R}^{n+1}$, not in~$\mathbb{R}^n$; the hypersphere below is~$S^n$, not $S^{n-1}$.
\end{note}

Dimensionality of a funcoid is defined as least dimensionality of a set above it.

Instead of proving the Kakeya conjecture, I will prove an obviously stronger statement:

\begin{claim}
(Funcoidal) join~$L$ of one-si\-de vicinities of each point of a set on~$\mathbb{R}^n$ has dimension~$n$, provided that the set~$K$ of directions of the vicinities is $n$-di\-men\-sio\-nal.
\end{claim}

Using that $\mathbb{R}^n$ is diffeomorphic to~$S^n_{>}$, we can instead prove the claim for~$S^n_{>}$:

\begin{claim}
(Funcoidal) join~$L$ of one-si\-de vicinities $v_a$ (laying on great circles) of each point on $S^n_{>}$ hemisphere has dimension~$n$, provided that the set~$K'$ of directions (measured as their ``projection'' to unit sphere) is $n$-di\-men\-sio\-nal.
\end{claim}

The claims are equivalent, because we can take~$K'$ (figure~\ref{fig:area}) as the result of the above considered diffeomorphism between~$\mathbb{R}^n$ and~$S^n_{>}$.

\begin{figure}[hbt]
    \centering
    \includegraphics[width=\textwidth]{area}
    \caption{The mapping of set of directions}
    \label{fig:area}
\end{figure}

Let's draw (see figure~\ref{fig:filters}) a point~$a_v$ through which passes a great circle with vicinty~$v$ ``inside'' that a plane directions is mapped to.

Replace each ``curved'' vicinity $v$ by a vicinity~$h_v$ in the tangent line of the great circle. Then take funcoidal join $H=\bigsqcup_{a\in K'}h_v$. Informally, $H$ is a ``swelling'' or ``hairs'' on the hypersphere in $\mathbb{R}^{n+1}$ into tangential directions.

\begin{figure}[hbt]
    \centering
    \includegraphics[width=\textwidth]{filters}
    \caption{The vicinities (infinitely short arrows)}
    \label{fig:filters}
\end{figure}

The vicinities $h_v$, projected to the sphere by the radial projection, become exactly original curved vicinities $v$: $\Lambda h_v=v$, where $\Lambda$ is the radial projection to the hypersphere.

Thus $L = \Lambda H$.

If $\Lambda H$ has dimension~$n$, then the claim is proved.

\begin{lem}
If a set $H$ has dimension~$n+1$, then its radial (toward the center of the hypersphere) projection~$K'$ to~$S^n$ has dimension~$n$.
\end{lem}

\begin{proof}
Follows from~\cite{189275}\footnote{Or \href{https://grok.com/share/bGVnYWN5_2e0aef8b-4309-420f-815c-a09d76ae97c1}{these AI musings}
that I didn't check for correctness.}.
(We also apply that spherical coordinates are diffeomorphic to Euclidean coordinates, except of a negligible set.)
\end{proof}

It follows, that if we prove that $H$ has dimension~$n+1$, then $L$~has dimension~$n$ and our claim is proved.

Let map (name the map $b$\footnote{From \textbf{b}lack hole.}) the sphere together with its swelling to coordinates center by the formula $r\mapsto r-1$ for $r\geq 1$ in spherical coordinates (leaving $\theta'$ the same).
This preserves the dimension because the Jacobian matrix has full rank almost everywhere (hint: The Jacobian is the identity matrix).
Obviously, it is a diffeomorphism of $\setcond{x}{r>1}$ to $\setcond{x}{r\ne 0}$.

Thus we have enough to prove that the join~$bH$ of one-si\-de vicinities~$w_v$ for a full-me\-a\-su\-re choice~$d$ of directions of the same point (the center of our sphere) is~$n$.

It's easy to show~(\cite{curve}) that
\[ b(c_v(t)) = \left(\sqrt{1+t^2}-1\right)\frac{a_v+tV_v}{\sqrt{1+t^2}} =
\left(1-\frac{1}{\sqrt{1+t^2}}\right)(a_v+tV_v). \]
where $c_v$~is the tangent line defined by $h_v$, $V_v$ is a unit vector of direction of vicinity~$v$.
After a reparametrization, \[ b_v'(t) = \left(1-\frac{1}{\sqrt{1+t^2}}\right)(a_v+tV_v). \]
Note that $b_v'(t)$ is curved (not straight line). Therefore $bh_v$ is a ``curved filter'' (filter laying inside a curve and not a straight line).

By properties of diffeomorphisms, $\dim bH = \dim H$.

\begin{lem}
$\dim bH<n+1$.
\end{lem}

\begin{proof}
Let $C$~be the Cantor set.

Solve?? $\norm{b_v'(1)\times V_v} = \sum_{k\in\mathbb{N}}q_k 3^{-k}$, where $\norm{a_v} = \sum_{k\in\mathbb{N}}\frac{q_k}{2} 2^{-k}$.

We have
\[ \norm{\left(1-\frac{1}{\sqrt{2}}\right)a_v\times V_v} = \sum_{k\in\mathbb{N}}q_k 3^{-k}. \]

Accordingly\fxwarning{Check the AI.}~\cite{zero-measure} the measure of $\setcond{b_a(t)}{a\in K, 0<t<1}$ is zero.
Therefore $\dim bH<n+1$.
\end{proof}

Therefore Kakeya conjecture is false.

\section{``Political'' context}

I discovered ordered semigroup/semicategory actions in 2019. The world didn't react obeying like a sheep to decision of Russian Orthodoxes to kick people like me from normal life such as academic carrier. That's scary: the most valuable component is a missing component. Missing ordered semigroup actions amount to like half of world economy missing~\cite{osa-important}.

I don't know how Kakeya conjecture is used in the rest of math, but I use it to build a bridge between two about halves: academic math and the second lost half, through my glorification.

The advice: ``Solve some famous open problem to confirm utility of your theory.'' sounded for me like a mocking. But in the new reality, I indeed did it.

\bibliographystyle{amsplain}
\bibliography{kakeya}

\end{document}