\documentclass[oneside,draft]{amsart}
\usepackage{amsmath,amssymb,amsthm,stmaryrd}
\usepackage[final]{hyperref}
\usepackage{xcolor}
\usepackage{fixme}
\fxsetup{draft}
\fxusetheme{color}

\newcommand{\setcond}[2]{\left\{#1\mid#2\right\}}
\newcommand{\intrs}{\not\asymp}
\newcommand{\nintrs}{\asymp}
\newcommand{\funcoids}{\mathsf{FCD}}
\newcommand{\atoms}{\operatorname{atoms}}
\newcommand{\supfun}[1]{\left\langle#1\right\rangle}
\newcommand{\suprel}[1]{\left[#1\right]}
% \newcommand{\del}[1]{\textcolor{gray}{#1}}
\newenvironment{del}{\color{gray}}{}

\newtheorem{thm}{Theorem}
\newtheorem{lem}{Lemma}
\newtheorem{obvious}{Obvious}
\newtheorem{cor}{Corollary}
\newtheorem{claim}{Claim}
\newtheorem{rem}{Remark}
\newtheorem{defn}{Definition}
\newtheorem{exer}{Exercise}

\title{Proof of Kakeya's conjecture}
\author{Victor Porton, ORCID 0000-0001-7064-7975}
\email{porton.victor@gmail.com}
\subjclass[2020]{28D99, 51M05, 54J99, 57N99}
\keywords{Kakeya conjecture, funcoids, dimensionality, diffeomorphism, geometry, general topology, long-standing open problem}

\begin{document}

\begin{abstract}
A rather short proof of Kakeya conjecture in an arbitrary Euclidean spaces is presented. The proof uses theory of $n$-fun\-co\-ids and several space transformations/projections. The conjecture is reduced to a Ka\-ke\-ya-li\-ke theorem on a hypersphere.
\end{abstract}

\maketitle  

\emph{This is a draft of my proof of Kakeya conjecture. Report any errors, please.}

The proof uses theory of funcoids~\cite{volume-1}. It looks like that funcoids can be easily excluded from the proof, making the proof elementary. In this case, the role of funcoids is to provide intuition that helped to find the proof.

\textbf{Kakeya set conjecture:} \cite{kakeya-long}~\cite{tao-blog-kakeya} Define a \emph{Besicovitch} set in $\mathbb{R}^n$ to be a set which contains a unit line segment in every direction. Is it true that such sets necessarily have Hausdorff dimension and Minkowski\footnote{Minkowski dimensions are not addressed in this draft.} dimension equal to~$n$?

\section{Kakeya proof}

\subsection{$n$-fun\-co\-ids}

By an $n$-fun\-co\-id (for a natural~$n$) we will understand a filter\footnote{Including the improper filter.} on the lattice~$\Gamma_n$ of finite unions of Cartesian products of $n$~sets. I will denote join and meet of funcoids in the order~$\sqsubseteq$ \emph{reverse} to set-the\-o\-re\-tic inclusion respectively as $\bigsqcup$ and $\bigsqcap$. ($n$-fun\-co\-ids form a complete lattice~\cite{volume-1}, because $\Gamma_n$~is a boolean lattice.) $\bot$~is the least element of the lattice.

I will also denote \[ a\nintrs b \Leftrightarrow \exists c\ne\bot: c\sqsubseteq a\land c\sqsubseteq b, \] or equivalently, $a\sqcap b\ne\bot$.
$a\intrs b\Leftrightarrow\lnot(a\nintrs b)$.

By $\mathord{\uparrow}F$ I denote the $n$-fun\-co\-id corresponding to $n$-ary relation~$F$ by the formula
\[ \mathord{\uparrow}F = \setcond{X\in\Gamma_n}{X\supseteq F}. \]

\begin{exer}
$\bigcap\mathord{\uparrow}F = F$.
\end{exer}

Thus, we have an injection from the set of $n$-ary relations (on some sets) to the set of $n$-fun\-co\-ids.
This allows to idenitify $F$ with $\mathord{\uparrow}F$.

For $n\leq m$ I will denote applying $m$-fun\-co\-id~$f$ to $n$-fun\-co\-id~$x$ as
\[ \bigsqcap\setcond{\mathord{\uparrow} F[X]}{F\in f,X\in x}, \]
where $F[X]$ is considered as a relation of $2$~arguments: $F\subseteq F_Y\times F_X$.

Note that $1$-fun\-co\-ids are the same as filters on a set.

\begin{defn}
$a\in\suprel{f} \Leftrightarrow \forall X\in f, i\in\{0,\dots,n-1\}: a_i \intrs \mathord{\uparrow}X_i$ for $f$~being an $n$-fun\-co\-id and $a$~being an indexed set of filters.
\end{defn}

\subsubsection{Product of funcoids}

\begin{defn}
Let $f_i = f_0,\dots,f_{k-1}$ be $n$-fun\-co\-ids (probably $n$~dependent on~$i$). The \emph{funcoidal product} $f\times g$ is defined by the formula:
\[
\prod^{\funcoids}f = \bigsqcap\setcond{\mathord{\uparrow}\prod F}{i\in\{0,\dots,n-1\}, F_i\in f_i}.
\]
\end{defn}

\begin{obvious}
$a_i\in\prod^{\funcoids}f \Leftrightarrow \forall i\in\{0,\dots,n-1\}: a_i\intrs f_i$.
\end{obvious}

\subsubsection{Atomic funcoids}

\begin{defn}
By \emph{atomic} $n$-fun\-co\-ids I understand atoms of the lattice of $n$-fun\-co\-ids.\footnote{I remind that the order considered is reverse to set-the\-o\-re\-tic inclusion, therefore atomic funcoids are the same as maximal filters on the lattice of funcoids.}
\end{defn}

\begin{thm}
Atoms of the lattice of funcoids are exactly funcoidal products of atomic filters\footnote{The same as ultrafilters.}.
\fxwarning{Also prove in the opposite direction.}
\end{thm}

\begin{proof}
A non-least funcoid~$f$ cannot be strictly below such a product~$p$, because otherwise by properties of generalized filter bases~\cite{volume-1} it would exist a Cartesian product of sets above~$p$ but not above~$f$, what is clearly impossible.
\end{proof}

\begin{obvious}
The lattice of $n$-fun\-co\-ids is atomic.
\end{obvious}

\begin{thm}
The lattice of $n$-fun\-co\-ids is atomistic (that is each $n$-fun\-co\-id is a join of atomic $n$-fun\-co\-ids).
\end{thm}

\begin{proof}
Similar to the proof for ($2$)-fun\-co\-ids in~\cite{volume-1}.
\end{proof}

\begin{thm}
For an indexed set~$A$ of sets, a relation~$\delta\subseteq(\prod_i\atoms A_i)$ such that (for every $a\in\prod\atoms A_i$)
\begin{multline*}
\forall X:(\forall i\in\{0,\dots,n-1\}:X_i\in a_i\Rightarrow\exists x\in\delta\forall i\in\{0,\dots,n-1\}, x_i\in\atoms\mathord{\uparrow}X_i)
\\\Rightarrow a\in\delta
\end{multline*}
can be continued to the relation $\suprel{f}$ for a unique $n$-fun\-co\-id~$f$;
\[ \mathcal{X}\in\suprel{f} \Leftrightarrow \exists x\in\delta\forall i\in\{0,\dots,n-1\}:x_i\in\atoms\mathcal{X}_i \]
for every filters $\mathcal{X}_i$ on $A_i$.
\end{thm}

\begin{proof}
\fxwarning{Prove.}
\end{proof}

\subsubsection{Rotating funcoids}

We need to rotate funcoids, because below in the proof we will consider vicinities of different directions.

The following theorem basically says that funcoids are invariant under rotations:

\begin{thm}
Let $P$ be a rotation of the space $\mathbb{R}^n$. Then
\[ (\lambda i\in\{0,\dots,n-1\}:P[x_i])\in\suprel{f'} \Leftrightarrow x\in\suprel{f} \]
for any filters~$x_0$, \dots, $x_{n-1}$ on $\mathbb{R}$,
where $f'$ is a unique $n$-fun\-co\-id in the space $P^{-1}[\mathbb{R}^n]$.
\end{thm}

\begin{proof}
\fxwarning{Prove. Probably, the easiest way to prove is through atomic funcoids.}
Without loss of generality (because every rotation can be represented as a composition of rotations in a plane) we can consider only a rotation in the plane of the axes~$i=0$ and~$i=1$. Let it be rotated by the formulas ??
\end{proof}

\fxwarning{Need the notation~$Pf$ for the rotated funcoid. This also extends to any motion~$P$ of the funcoid, because any motion is composition of shifts (obviously preserving the properties of a funcoid) and rotations (preserving by the above theorem).}
\fxwarning{More theorems about invariance of funcoids under rotations?}

\subsection{Spherical coordinates}

I will denote $n$-dimensional spherical coordinates as~$\theta=(r, \theta_1,\dots,\theta_{n-1})$.
I also denote direction $\theta'=(\theta_1,\dots,\theta_{n-1})$.

Here $r\geq 0$; $\theta_i\in[0;\pi]$ for $i=1,\dots,n-2$; $\theta_{n-1}\in[0;2\pi\mathclose[$ .

Mapping $\Sigma_n$ from spherical coordinates to Euclidean coordinates is defined by the formulas:
\[
\begin{aligned}x_{1}&=r\cos(\theta _{1}),\\x_{2}&=r\sin(\theta _{1})\cos(\theta _{2}),\\x_{3}&=r\sin(\theta _{1})\sin(\theta _{2})\cos(\theta _{3}),\\&\qquad \vdots \\x_{n-1}&=r\sin(\theta _{1})\cdots \sin(\theta _{n-2})\cos(\theta _{n-1}),\\x_{n}&=r\sin(\theta _{1})\cdots \sin(\theta _{n-2})\sin(\theta _{n-1}).\end{aligned}
\]

Accordingly~\cite{polar-jacobian} the determinant of Jacobian of this transformation is
\[ \det (J\Sigma_n)(\theta) = r\sin\theta_1\dots\sin\theta_{n-2}\det (J\Sigma_{n-1})(\theta). \]
Therefore
\[
\det (J\Sigma_n)(\theta)\ne 0\Leftrightarrow 0\notin\{r,\sin\theta_1,\dots,\sin\theta_{n-2}\}.
\]
So, $\det (J\Sigma_n)(\theta)\ne 0$ almost everywhere.

Mapping $\Pi_n$ from Euclidean coordinates to spherical coordinates is defined by the formulas:
\[
\begin{aligned}r&={\textstyle {\sqrt {{x_{n}}^{2}+{x_{n-1}}^{2}+\cdots +{x_{2}}^{2}+{x_{1}}^{2}}}},\\\theta _{1}&=\operatorname {atan2} \left({\textstyle {\sqrt {{x_{n}}^{2}+{x_{n-1}}^{2}+\cdots +{x_{2}}^{2}}}},x_{1}\right),\\\theta _{2}&=\operatorname {atan2} \left({\textstyle {\sqrt {{x_{n}}^{2}+{x_{n-1}}^{2}+\cdots +{x_{3}}^{2}}}},x_{2}\right),\\&\qquad \vdots \\\theta _{n-2}&=\operatorname {atan2} \left({\textstyle {\sqrt {{x_{n}}^{2}+{x_{n-1}}^{2}}}},x_{n-2}\right),\\\theta _{n-1}&=\operatorname {atan2} \left(x_{n},x_{n-1}\right).\end{aligned}
\]  

At points $0\notin\{r,\sin\theta_1,\dots,\sin\theta_{n-2}\}$ that is $0\notin\{x_1,\dots,x_{n-1}\}$ the Jacobian $(J\Pi_n)(x)$
is full rank as a reverse of a full rank Jacobian.

So, $\det (J\Pi_n)(x)\ne 0$ almost everywhere.

\subsection{Stereographic projection diffeomorphism}

Consider~$\Gamma_n$, the ``standard'' stereographic projection from $S^n$ without the North pole $\zeta=(0,\dots,0,1)$ to $\mathbb{R}^n$: \fxwarning{Duplicate notation with the section about funcoids.}

\begin{align*}
&\Gamma_n(q_1,\dots,q_{n+1}) = \left(\frac{q_1}{1-q_{n+1}},\dots,\frac{q_{n-1}}{1-q_{n+1}}\right);\\
&\Gamma_n^{-1}(x_1,\dots,x_n) = \left(\frac{2x_1}{\lVert x\rVert^2+1},\dots,\frac{2x_{n+1}}{\lVert x\rVert^2+1},\frac{\lVert x\rVert^2-1}{\lVert x\rVert^2+1}\right),
\end{align*}
where $\lVert x\rVert^2 = x_1^2+\dots+x_n^2$.

It is known~\cite{stereo} to be a diffeomorphism.

\subsection{Gnomonic projection}

Gnomonic projection is like stereographic projection, but with the projection hyperplane passing through the center of the hypersphere.

\fxwarning{Below were copied from an AI, check.}

The formulas for projection of a point $p=(p_1,\dots,p_{n+1})$ ($\lVert p\rVert = 1$) of the hypersphere are:

$q=\frac{p}{s\cdot p}$, where $q$ is the projection, $s$ is the point of the sphere. Note that $q_{n+1}=1$. We will exclude this coordinate.

It is evident \fxwarning{Prove.} that gnomonic projection is a diffeomorphism between the hyperplane and ``open-half-hy\-per\-sphe\-re'' ($s\cdot p>0$). Let denote it $S^n_{>}$.

I will denote the Gnomonic projection as~$\Phi_n$.

It is well known that Gnomonic projection maps great circles to lines.

\subsection{The proof}

The intuition how I came up with this solution: I tried to apply funcoids theory to an open problem. Thus I replaced finite intervals in the problem formulation by funcoids (particularly, by one-si\-de vicinities of a point). I wanted to take a product of all the funcoids to get an $n$-di\-men\-si\-on\-al set, but found it not making sense, because different vicinities being attached to different points. Then I decided to transform the space in such a way to make all points into one. The transformation I intended was decreasing spherical coordinate~$r$ by one, with the problem on the sphere instead of the plane. But this way, I lost all the structure. So, I decided to move the vicinities outside the sphere. I also added radial vicinities, to be sure it fills the entire space. I reduced $n$-di\-men\-sio\-na\-li\-ty of the set of Kakeya set to $(n+1)$-di\-men\-sio\-na\-li\-ty of the resulting ``swollen'' sphere.

Let $K$ be a Besicovitch set in $\mathbb{R}^n$.

I will call the \emph{right vicinity} or \emph{one-si\-de vicinity} of a point~$a$ the funcoid~$\supfun{\Delta_{\geq}}\{a\}$.

We will do our geometry in $\mathbb{R}^{n+1}$.

Dimensionality of a funcoid is defined as least dimensionality of a set above it.

Instead of proving the Kakeya conjecture, I will prove a stronger statement:

\begin{claim}
(Funcoidal) join~$L$ of one-si\-de vicinities $v_a$ (laying on great circles) of each point on $S^n_{>}$ hemisphere has dimension~$n$, provided that the set~$K_1$ of directions (measured as their ``projection'' to unit sphere) is $n$-di\-men\-sio\-nal.
\end{claim}

Replace each ``curved'' vicinity $v_a$ by a straight one~$h_a$ (going outside the hypersphere in~$\mathbb{R}^{n+1}$) in the tangential ray determined by~$v_a$. Let $s_a$ be an ``outward'' radial (orthogonal to the hypersphere at point~$a$) one-si\-de vicinity. Then take funcoidal join $H=\bigsqcup_{a\in K_1}(h_a\sqcup s_a)$. Informally, $H$ is a ``swelling'' or ``hairs'' on the hypersphere in $\mathbb{R}^{n+1}$ into both tangential and orthogonal directions.

The straight vicinities $h_a$, projected to the sphere by the radial projection, become exactly original curved vicinities $v_a$ (because they lay on great circles): $\Lambda h_a=v_a$, where $\Lambda$ is the radial projection to the hypersphere.

Thus $L = \Phi_n H$.

If $\Phi_n H$ has dimension~$n$, then the claim is proved.

\begin{lem}
If a set $H$ has dimension~$n+1$, then its radial (toward the center of the hypersphere) projection~$K_1$ to~$S^n$ has dimension~$n$.
\end{lem}

\begin{proof}
Follows from~\cite{189275}\footnote{Or \href{https://grok.com/share/bGVnYWN5_2e0aef8b-4309-420f-815c-a09d76ae97c1}{these AI musings}
that I didn't check for correctness.}.
(We also apply that spherical coordinates are diffeomorphic to Euclidean coordinates, except of a negligible set.)
\end{proof}

It follows, that if we prove that $H$ has dimension~$n+1$, then $K_1$~has dimension~$n$ and our claim is proved.

Let map (name the map $b$\footnote{From \textbf{b}lack hole.}) the sphere together with its swelling to coordinates center by the formula $r\mapsto r-1$ for $r\geq 1$ in spherical coordinates (leaving $\theta'$ the same).
This preserves the dimension because the Jacobian matrix has full rank almost everywhere (hint: The Jacobian is the identity matrix).
Obviously, it is a diffeomorphism of $\setcond{x}{r>1}$ to $\setcond{x}{r\ne 0}$.

\begin{obvious}
$\supfun{b}s_a$ is a vicinity from the center to the direction~$a$.
\end{obvious}

Thus we have enough to prove that the join~$bH$ of one-si\-de vicinities~$w_d$ for a full-me\-a\-su\-re choice~$d$ of directions of the same point (the center of our sphere) is~$n$.

By properties of diffeomorphisms, $\dim bH = \dim H$.

By properties of $n$-fun\-co\-ids, it's evident (because it, with possible exclusion of a finite set, is mapped by~$\Pi_n$ to a funcoidal product of full-me\-a\-su\-re one-si\-de vicinities on~$\mathbb{R}$) that $\dim bH=n+1$.

\section{``Political'' context}

I discovered ordered semigroup/semicategory actions in 2019. The world didn't react obeying like a sheep to decision of Russian Orthodoxes to kick people like me from normal life such as academic carrier. That's scary: the most valuable component is a missing component. Missing ordered semigroup actions amount to like half of world economy missing~\cite{osa-important}.

I don't know how Kakeya conjecture is used in the rest of math, but I use it to build a bridge between two about halves: academic math and the second lost half, through my glorification.

The advice: ``Solve some famous open problem to confirm utility of your theory'' sounded for me like a mocking. But in the new reality, I indeed did it.

\bibliographystyle{amsplain}
\bibliography{kakeya}

\end{document}