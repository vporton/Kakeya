\documentclass{amsart}
\usepackage{amsmath,amssymb,amsthm}
\usepackage{hyperref}
\usepackage{xcolor}

\newcommand{\setcond}[2]{\left\{#1\mid#2\right\}}
\newcommand{\funcoids}{\mathsf{FCD}}
\newcommand{\supfun}[1]{\left\langle#1\right\rangle}
% \newcommand{\del}[1]{\textcolor{gray}{#1}}
\newenvironment{del}{\color{gray}}{}

\newtheorem{theorem}{Theorem}
\newtheorem{lem}{Lemma}
\newtheorem{claim}{Claim}
\newtheorem{rem}{Remark}

\title{Proof of Kakeya's conjecture}
\author{Victor Porton, ORCID 0000-0001-7064-7975}
\email{porton.victor@gmail.com}

\begin{document}

\maketitle  

\emph{This is a preliminary draft sketch of my proposed proof of Kakeya conjecture. Report any errors, please.}

The proof uses theory of funcoids~\cite{volume-1}. It looks like that funcoids can be easily excluded from the proof, making the proof elementary. In this case, the role of funcoids is to provide intuition that helped to find the proof.

\textbf{Kakeya set conjecture:} Define a \emph{Besicovitch} set in $\mathbb{R}^n$ to be a set which contains a unit line segment in every direction. Is it true that such sets necessarily have Hausdorff dimension and Minkowski dimension equal to~$n$?

\section{Kakeya proof (rough draft)}

\subsection{Spherical coordinates}

I will denote $n$-dimensional spherical coordinates as~$\theta=(r, \theta_1,\dots,\theta_{n-1})$.
I also denote direction $\theta'=(\theta_1,\dots,\theta_{n-1})$.

Here $r\geq 0$; $\theta_i\in[0;\pi]$ for $i=1,\dots,n-2$; $\theta_{n-1}\in[0;2\pi\mathclose[$ .

Mapping $\Sigma_n$ from spherical coordinates to Euclidean coordinates is defined by the formulas:
\[
\begin{aligned}x_{1}&=r\cos(\theta _{1}),\\x_{2}&=r\sin(\theta _{1})\cos(\theta _{2}),\\x_{3}&=r\sin(\theta _{1})\sin(\theta _{2})\cos(\theta _{3}),\\&\qquad \vdots \\x_{n-1}&=r\sin(\theta _{1})\cdots \sin(\theta _{n-2})\cos(\theta _{n-1}),\\x_{n}&=r\sin(\theta _{1})\cdots \sin(\theta _{n-2})\sin(\theta _{n-1}).\end{aligned}
\]

Accordingly~\cite{polar-jacobian} the determinant of Jacobian of this transformation is
\[ \det (J\Sigma_n)(\theta) = r\sin\theta_1\dots\sin\theta_{n-2}\det (J\Sigma_{n-1})(\theta). \]
Therefore
\[
\det (J\Sigma_n)(\theta)\ne 0\Leftrightarrow 0\notin\{r,\sin\theta_1,\dots,\sin\theta_{n-2}\}.
\]
So, $\det (J\Sigma_n)(\theta)\ne 0$ almost everywhere.

Mapping $\Pi_n$ from Euclidean coordinates to spherical coordinates is defined by the formulas:
\[
\begin{aligned}r&={\textstyle {\sqrt {{x_{n}}^{2}+{x_{n-1}}^{2}+\cdots +{x_{2}}^{2}+{x_{1}}^{2}}}},\\\theta _{1}&=\operatorname {atan2} \left({\textstyle {\sqrt {{x_{n}}^{2}+{x_{n-1}}^{2}+\cdots +{x_{2}}^{2}}}},x_{1}\right),\\\theta _{2}&=\operatorname {atan2} \left({\textstyle {\sqrt {{x_{n}}^{2}+{x_{n-1}}^{2}+\cdots +{x_{3}}^{2}}}},x_{2}\right),\\&\qquad \vdots \\\theta _{n-2}&=\operatorname {atan2} \left({\textstyle {\sqrt {{x_{n}}^{2}+{x_{n-1}}^{2}}}},x_{n-2}\right),\\\theta _{n-1}&=\operatorname {atan2} \left(x_{n},x_{n-1}\right).\end{aligned}
\]  

At points $0\notin\{r,\sin\theta_1,\dots,\sin\theta_{n-2}\}$ that is $0\notin\{x_1,\dots,x_{n-1}\}$ the Jacobian $(J\Pi_n)(x)$
is full rank as a reverse of a full rank Jacobian.

So, $\det (J\Pi_n)(x)\ne 0$ almost everywhere.

\subsection{Stereographic projection diffeomorphism}

Consider~$\Gamma_n$, the ``standard'' stereographic projection from $S^n$ without the North pole $\zeta=(0,\dots,0,1)$ to $\mathbb{R}^n$:

\begin{align*}
&\Gamma_n(q_1,\dots,q_{n+1}) = \left(\frac{q_1}{1-q_{n+1}},\dots,\frac{q_{n-1}}{1-q_{n+1}}\right);\\
&\Gamma_n^{-1}(x_1,\dots,x_n) = \left(\frac{2x_1}{\lVert x\rVert^2+1},\dots,\frac{2x_{n+1}}{\lVert x\rVert^2+1},\frac{\lVert x\rVert^2-1}{\lVert x\rVert^2+1}\right),
\end{align*}
where $\lVert x\rVert^2 = x_1^2+\dots+x_n^2$.

It is known [TODO: source] to be a diffeomorphism.

\subsection{The proof}

Let $K$ be a Besicovitch set in $\mathbb{R}^n$.

I will call the \emph{right vicinity} or \emph{one-si\-de vicinity} of a point~$a$ the filter~$\supfun{\Delta_{>}}\{a\}$.

We will do our geometry in $\mathbb{R}^{n+1}$.

Instead of proving the Kakeya conjecture, we will prove a stronger statement:

\begin{claim}
(Funcoidal) join~$L$ of one-si\-de vicinities of each point on $S^n\setminus\{\zeta\}$ has dimension~$n$, provided that the set~$K_1$ of directions (measured as their ``projection'' to unit sphere) is $n$-di\-men\-sio\-nal. [FIXME: North pole?]
\end{claim}

(Here dimension of a funcoid is defined as least dimension of a set above it.)

\begin{rem}
The set of projections of vicinities has the same dimensionality as the set of directions, because of the diffeomorphism.
\end{rem}

\begin{del}
Consider a diffeomorphism of an open set of the sphere to open disk with center of the point~$a$ of touching in each its tangential hyperplane.

Consider vicinities from the sphere replaced by vicinities from the disk (we may [TODO] further distort the disk in order to make them straight).

Join of all vicinities in the disk, is some filter under the disk. Infinitely shrinking the disk, we get its corresponding funcoidal meet~$h_a$. [TODO: Funcoid or reloid?]
\end{del}

Replace each ``curved'' vicinity from point~$a$ of the hypersphere by a straight one~$h_a$ (going outside the hypersphere in~$\mathbb{R}^{n+1}$). Then take funcoidal join $H=\bigsqcup_{a\in K_1}h_a$ of all these straight vicinities. $H$ is a ``swollen'' hypersphere in $\mathbb{R}^{n+1}$.

If $H$ projected to the sphere (by straight lines toward its center) has dimension~$n$, then the claim is proved, because it is [TODO: prove] diffeomorphic to~$L$.

\begin{lem}
If a set $H$ has dimension~$n+1$, then its projection~$K_1$ to~$S^n$ has dimension~$n$.
\end{lem}

\begin{proof}
Suppose, $\dim K_1<n$. Then \[ \dim H\leq\dim\setcond{tx}{t\geq 0,x\in K_1} =?? \dim K_1+1<n+1, \] contradiction.
\end{proof}

If [why?] we prove that $H$ has dimension~$n+1$, then then $K_1$~has dimension~$n$ and our claim is proved.

Let map the sphere together with its swelling to coordinates center by the formula $r\mapsto r-1$ for $r\geq 1$ (leaving $\theta'$ the same).
This preserves the dimension because the Jacobian matrix has full rank almost everywhere (hint: The Jacobian is the identity matrix).
Obviously, it is a diffeomorphism of $\setcond{x}{r>1}$ to $\setcond{x}{r\ne 0}$. So we preserve the dimension of the filter~$H$.

Thus we have enough to prove that join of one-si\-de vicinities of the same point (the center of our sphere without the North pole) is~$n$.

\begin{lem}
$h_a$ is projected to a vicinity from the center to the direction~$a$.
\end{lem}

\begin{proof}
$h_a$ is a vicinity of the point~$a$ on some line tangential to the hypersphere, immerse it into an interval $\left]0;\epsilon\right[$. This $0$ is projected to the center of the sphere, $\epsilon$ is projected approximately to the direction~$a$. Taking the funcoidal meet of all such intervals, we get a vicinity from the center to the direction~$a$. [FIXME: Is the resulting vicinity on a straight line?]
\end{proof}

By properties of diffeomorphisms, the shrinked by $r\mapsto r-1$ swelling~$H$ has the same dimensionality as~$H$.

By properties of funcoids, it's evident that after being shrinked, it has dimensionality~$n+1$.

\begin{del}
To prove this use di\-men\-si\-on-pre\-ser\-ving map from spherical coordinates to Euclidean coordinates.

We've got the funcoidal product (e.g.\ subatomic product) of $n$\ vicinities what is obviously of dimension~$n$.
\end{del}

\bibliographystyle{amsplain}
\bibliography{kakeya}

\end{document}