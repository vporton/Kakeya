\documentclass[oneside,draft]{amsart}
\usepackage{amsmath,amssymb,amsthm,stmaryrd}
\usepackage[final]{hyperref}
\usepackage[final]{graphicx}
\usepackage{xcolor}
\usepackage{fixme}
\fxsetup{draft}
\fxusetheme{color}

\newcommand{\setcond}[2]{\left\{#1\mid#2\right\}}
\newcommand{\intrs}{\not\asymp}
\newcommand{\nintrs}{\asymp}
\newcommand{\funcoids}{\mathsf{FCD}}
\newcommand{\atoms}{\operatorname{atoms}}
\newcommand{\norm}[1]{\lVert #1\rVert}
\newcommand{\supfun}[1]{\left\langle#1\right\rangle}
\newcommand{\suprel}[1]{\left[#1\right]}
% \newcommand{\del}[1]{\textcolor{gray}{#1}}
\newenvironment{del}{\color{gray}}{}

\newtheorem{thm}{Theorem}
\newtheorem{lem}{Lemma}
\newtheorem{prop}{Proposition}
\newtheorem{note}{Note}
\newtheorem{obvious}{Obvious}
\newtheorem{cor}{Corollary}
\newtheorem{claim}{Claim}
\newtheorem{rem}{Remark}
\newtheorem{defn}{Definition}
\newtheorem{exer}{Exercise}

\title{A Proof of Kakeya's Conjecture}
\author{Victor Porton, ORCID 0000-0001-7064-7975}
\email{porton.victor@gmail.com}
\subjclass[2020]{28D99, 51M05, 54J99, 57N99, 51M15, 51M99}
\keywords{Kakeya conjecture, funcoids, dimensionality, diffeomorphism, geometry, general topology, long-standing open problem}

\begin{document}

\begin{abstract}
A rather short proof of Kakeya conjecture in arbitrary Euclidean spaces is presented. The proof uses several space transformations/projections. The conjecture is reduced to a Ka\-ke\-ya-li\-ke theorem on a hypersphere.
\end{abstract}

\maketitle  

In this article, I prove Kakeya conjecture in an arbitrary Euclidean space.

\textbf{Kakeya set conjecture:} \cite{kakeya-long,tao-blog-kakeya} Define a \emph{Besicovitch} set in $\mathbb{R}^n$ to be a set which contains a unit line segment in every direction. Is it true that such sets necessarily have Hausdorff dimension and Minkowski dimension equal to~$n$?

We will prove only for Hausdorff dimension, because Minkowski dimension cannot be lower and having a proof for Hausdorff dimension also proves it for Minkowski one. $\dim$ operator will mean Hausdorff dimension.

\section{Kakeya proof}

\subsection{Spherical coordinates}

I will denote $n$-dimensional spherical coordinates as~$\theta=(r, \theta_1,\dots,\theta_{n-1})$.
I also denote direction $\theta'=(\theta_1,\dots,\theta_{n-1})$.

Here $r\geq 0$; $\theta_i\in[0;\pi]$ for $i=1,\dots,n-2$; $\theta_{n-1}\in[0;2\pi\mathclose[$ .

Mapping $\Sigma_n$ from spherical coordinates to Euclidean coordinates is defined by the formulas:
\[
\begin{aligned}x_{1}&=r\cos(\theta _{1}),\\x_{2}&=r\sin(\theta _{1})\cos(\theta _{2}),\\x_{3}&=r\sin(\theta _{1})\sin(\theta _{2})\cos(\theta _{3}),\\&\qquad \vdots \\x_{n-1}&=r\sin(\theta _{1})\cdots \sin(\theta _{n-2})\cos(\theta _{n-1}),\\x_{n}&=r\sin(\theta _{1})\cdots \sin(\theta _{n-2})\sin(\theta _{n-1}).\end{aligned}
\]

Accordingly~\cite{polar-jacobian} the determinant of Jacobian of this transformation is
\[ \det (J\Sigma_n)(\theta) = r\sin\theta_1\dots\sin\theta_{n-2}\det (J\Sigma_{n-1})(\theta). \]
Therefore
\[
\det (J\Sigma_n)(\theta)\ne 0\Leftrightarrow 0\notin\{r,\sin\theta_1,\dots,\sin\theta_{n-2}\}.
\]
So, $\det (J\Sigma_n)(\theta)\ne 0$ almost everywhere.

Mapping $\Pi_n$ from Euclidean coordinates to spherical coordinates is defined by the formulas:
\[
\begin{aligned}r&={\textstyle {\sqrt {{x_{n}}^{2}+{x_{n-1}}^{2}+\cdots +{x_{2}}^{2}+{x_{1}}^{2}}}},\\\theta _{1}&=\operatorname {atan2} \left({\textstyle {\sqrt {{x_{n}}^{2}+{x_{n-1}}^{2}+\cdots +{x_{2}}^{2}}}},x_{1}\right),\\\theta _{2}&=\operatorname {atan2} \left({\textstyle {\sqrt {{x_{n}}^{2}+{x_{n-1}}^{2}+\cdots +{x_{3}}^{2}}}},x_{2}\right),\\&\qquad \vdots \\\theta _{n-2}&=\operatorname {atan2} \left({\textstyle {\sqrt {{x_{n}}^{2}+{x_{n-1}}^{2}}}},x_{n-2}\right),\\\theta _{n-1}&=\operatorname {atan2} \left(x_{n},x_{n-1}\right).\end{aligned}
\]  

At points $0\notin\{r,\sin\theta_1,\dots,\sin\theta_{n-2}\}$ that is $0\notin\{x_1,\dots,x_{n-1}\}$ the Jacobian $(J\Pi_n)(x)$
is full rank as a reverse of a full rank Jacobian.

So, $\det (J\Pi_n)(x)\ne 0$ almost everywhere.

\subsection{Gnomonic projection}

Gnomonic projection (figure~\ref{fig:gnomonic}) is a projection with the center of the hypersphere being the center of the projection and the projection hyperplane not passing through the center of the sphere.

\begin{figure}[hbt]
    \centering
    \includegraphics[width=\textwidth]{gnomonic}
    \caption{Gnomonic projection}
    \label{fig:gnomonic}
\end{figure}

The formulas for projection of a point $p=(p_1,\dots,p_{n+1})$ ($\norm{p} = 1$) of the hypersphere are:

$q=\frac{p}{s\cdot p}$, where $q$ is the projection, $s$ is the North point of the sphere. Note that $q_{n+1}=1$. We will exclude this coordinate.

It is evident that gnomonic projection is a diffeomorphism between the hyperplane and ``open-half-hy\-per\-sphe\-re'' ($s\cdot p>0$). Let denote it~$S^n_{>}$.

I will denote the Gnomonic projection as~$\Phi_n$.

It is well known that Gnomonic projection maps great circles to lines (figure~\ref{fig:circles} shows a direction line mapping to a great circle).

\begin{figure}[hbt]
    \centering
    \includegraphics[width=\textwidth]{circles}
    \caption{A great circle corresponding to line direction}
    \label{fig:circles}
\end{figure}

\subsection{The proof}

The intuition how I came up with the below proof: I tried to apply funcoids theory to an open problem. Thus I replaced finite intervals in the problem formulation by funcoids (particularly, by one-si\-de vicinities of a point). I wanted to take a product of all the funcoids to get an $n$-di\-men\-si\-on\-al set, but found it not making sense, because different vicinities being attached to different points. Then I decided to transform the space in such a way to make all points into one. The transformation I intended was decreasing spherical coordinate~$r$ by one, with the problem on the sphere instead of the plane. But this way, I lost all the structure. So, I decided to move the vicinities outside the sphere. I added tangential vicinities, to be sure it fills the entire space. I reduced $n$-di\-men\-sio\-na\-li\-ty of the set of Kakeya set to $(n+1)$-di\-men\-sio\-na\-li\-ty of the resulting ``swollen'' sphere. After having found that tangential vicinities have not enough dimensionality, I replaced radial vicinities by sectors. Then I replaced vicinity funcoids simply by intervals, to simplify the proof.

Let $K$ be a Besicovitch set in $\mathbb{R}^n$.

We will do our geometry in $\mathbb{R}^{n+1}$.

\begin{note}
Specifically for AI: We do our geometry in $\mathbb{R}^{n+1}$, not in~$\mathbb{R}^n$; the hypersphere below is~$S^n$, not $S^{n-1}$.
\end{note}

Instead of proving the Kakeya conjecture, I will prove an obviously stronger statement:

\begin{claim}
Union~$L$ of intervals in~$\mathbb{R}^n$ has dimension~$n$, provided that the set~$K$ of directions of the intervals is $n$-di\-men\-sio\-nal.
\end{claim}

Using that $\mathbb{R}^n$ is diffeomorphic to~$S^n_{>}$, we can instead prove the claim for~$S^n_{>}$:

\begin{claim}
Union~$L$ of arcs $v$ in~$S^n_{>}$ hemisphere has dimension~$n$, provided that the set~$K'$ of directions of the arcs is $n$-di\-men\-sio\-nal.
\end{claim}

The claims are equivalent, because we can take~$K'$ (figure~\ref{fig:area}) as the result of the above considered diffeomorphism between~$\mathbb{R}^n$ and~$S^n_{>}$.

\begin{figure}[hbt]
    \centering
    \includegraphics[width=\textwidth]{area}
    \caption{The mapping of set of directions}
    \label{fig:area}
\end{figure}

Let's draw (see figure~\ref{fig:filters}) an arc~$v$ of the unit hypersphere.

Let $z$ be projection to to the circle toward its center: $z(p)=\frac{p}{\norm{p}}$.

Replace each arc $v$ by the set~$h_v=z^{-1}[v]$  (a angular region in a plane). Then take the union $H=\bigcup_{a\in K'}h_v$.

\begin{figure}[hbt]
    \centering
    \includegraphics[width=\textwidth]{filters}
    \caption{The polar beast}
    \label{fig:filters}
\end{figure}

The sets $h_v$, projected to the sphere by the radial projection, become exactly original arcs $v$: $\Lambda h_v=v$, where $\Lambda$ is the radial projection to the hypersphere.

Thus $L = \Lambda[H]$.

If $\dim\Lambda[H]=n$, then the claim is proved.

\begin{prop}
$\dim(P\times\mathbb{R}^n) = \dim P + n$ for any set $P$.
\end{prop}

\begin{proof}
\cite{189275}\footnote{Or \href{https://grok.com/share/bGVnYWN5_2e0aef8b-4309-420f-815c-a09d76ae97c1}{these AI musings}
that I didn't check for correctness.}.
\end{proof}

\begin{cor}
If a set $H$ has dimension~$n+1$, then its radial (toward the center of the hypersphere) projection~$K'$ to~$S^n$ has dimension~$n$.
\end{cor}

\begin{proof}
Apply to~$K'$ the reverse of the projection function~$z^{-1}$. The result is $H'=z^{-1}[K']\supseteq H$ and therefore $\dim(K'\times\mathbb{R})=\dim H'=n+1$. Therefore $\dim K'=n$.
We also apply that spherical coordinates are diffeomorphic to Euclidean coordinates, except of a negligible set.
\end{proof}

It follows, that if we prove that $H$ has dimension~$n+1$, then $L$~has dimension~$n$ and our claim is proved.

\fxerror{This mistakenly assumes that directions of great circles on~$S^n$ correspond to directions of intervals on~$\mathbb{R}^n$.}
But $H$ has dimension~$n+1$, because it is diffeomorphic to the Cartesian product a full-me\-a\-su\-re set of arcs on great circles and $\mathbb{R}$.

??I will try to prove that if the planar set of intervals has full-me\-a\-su\-re set of directions, then the corresponding arcs are also full-me\-a\-su\-re.

??We can?? also exchange the position and the direction on the hypersphere. Does this employ projective geometry?

\section{``Political'' context}

I discovered ordered semigroup/semicategory actions in 2019. The world didn't react obeying like a sheep to decision of Russian Orthodoxes to kick people like me from normal life such as academic carrier. That's scary: the most valuable component is a missing component. Missing ordered semigroup actions amount to like half of world economy missing~\cite{osa-important}.

I don't know how Kakeya conjecture is used in the rest of math, but I use it to build a bridge between two about halves: academic math and the second lost half, through my glorification.

The advice: ``Solve some famous open problem to confirm utility of your theory.'' sounded for me like a mocking. But in the new reality, I indeed did it.

\bibliographystyle{amsplain}
\bibliography{kakeya}

\end{document}