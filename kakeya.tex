\documentclass[oneside]{amsart}
\usepackage{amsmath,amssymb,amsthm,stmaryrd}
\usepackage{hyperref}
\usepackage{xcolor}
\usepackage{fixme}
\fxsetup{draft}
\fxusetheme{color}

\newcommand{\setcond}[2]{\left\{#1\mid#2\right\}}
\newcommand{\funcoids}{\mathsf{FCD}}
\newcommand{\supfun}[1]{\left\langle#1\right\rangle}
% \newcommand{\del}[1]{\textcolor{gray}{#1}}
\newenvironment{del}{\color{gray}}{}

\newtheorem{theorem}{Theorem}
\newtheorem{lem}{Lemma}
\newtheorem{obvious}{Obvious}
\newtheorem{cor}{Corollary}
\newtheorem{claim}{Claim}
\newtheorem{rem}{Remark}
\newtheorem{exer}{Exercise}

\title{Proof of Kakeya's conjecture}
\author{Victor Porton, ORCID 0000-0001-7064-7975}
\email{porton.victor@gmail.com}

\begin{document}

\maketitle  

\emph{This is a draft of my proof of Kakeya conjecture. Report any errors, please.}

The proof uses theory of funcoids~\cite{volume-1}. It looks like that funcoids can be easily excluded from the proof, making the proof elementary. In this case, the role of funcoids is to provide intuition that helped to find the proof.

\textbf{Kakeya set conjecture:} \cite{kakeya-long}~\cite{tao-blog-kakeya} Define a \emph{Besicovitch} set in $\mathbb{R}^n$ to be a set which contains a unit line segment in every direction. Is it true that such sets necessarily have Hausdorff dimension and Minkowski\footnote{Minkowski dimensions are not addressed in this draft.} dimension equal to~$n$?

\section{Kakeya proof}

\subsection{$n$-fun\-co\-ids}

By an $n$-fun\-co\-id (for a natural~$n$) we will understand a filter on the set~$\Gamma_n$ of finite unions of Cartesian products of $n$~sets. I will denote join and meet of funcoids in the order \emph{reverse} to set-the\-o\-re\-tic inclusion respectively as $\bigsqcup$ and $\bigsqcap$.

By $\mathord{\uparrow}F$ I denote the $n$-fun\-co\-id corresponding to $n$-ary relation~$F$ by the formula
\[ \mathord{\uparrow}F = \setcond{X\in\Gamma_n}{X\supseteq F}. \]

\begin{exer}
$\bigcap\mathord{\uparrow}F = F$.
\end{exer}

Thus, we have an injection from the set of $n$-ary relations (on some sets) to the set of $n$-fun\-co\-ids.
This allows to idenitify $F$ with $\mathord{\uparrow}F$.

For $n\leq m$ I will denote applying $m$-fun\-co\-id~$f$ to $n$-fun\-co\-id~$x$ as
\[ \bigsqcap\setcond{\mathord{\uparrow} F[X]}{F\in f,X\in x}, \]
where $F[X]$ is considered as a relation of $2$~arguments: $F\subseteq F_Y\times F_X$.

\fxwarning{Need to prove that $n$-fun\-co\-ids are invariant regarding rotations. This is used to rotate the vicinities.
Probably, the easiest way to prove is through atomic funcoids.}

\subsection{Spherical coordinates}

I will denote $n$-dimensional spherical coordinates as~$\theta=(r, \theta_1,\dots,\theta_{n-1})$.
I also denote direction $\theta'=(\theta_1,\dots,\theta_{n-1})$.

Here $r\geq 0$; $\theta_i\in[0;\pi]$ for $i=1,\dots,n-2$; $\theta_{n-1}\in[0;2\pi\mathclose[$ .

Mapping $\Sigma_n$ from spherical coordinates to Euclidean coordinates is defined by the formulas:
\[
\begin{aligned}x_{1}&=r\cos(\theta _{1}),\\x_{2}&=r\sin(\theta _{1})\cos(\theta _{2}),\\x_{3}&=r\sin(\theta _{1})\sin(\theta _{2})\cos(\theta _{3}),\\&\qquad \vdots \\x_{n-1}&=r\sin(\theta _{1})\cdots \sin(\theta _{n-2})\cos(\theta _{n-1}),\\x_{n}&=r\sin(\theta _{1})\cdots \sin(\theta _{n-2})\sin(\theta _{n-1}).\end{aligned}
\]

Accordingly~\cite{polar-jacobian} the determinant of Jacobian of this transformation is
\[ \det (J\Sigma_n)(\theta) = r\sin\theta_1\dots\sin\theta_{n-2}\det (J\Sigma_{n-1})(\theta). \]
Therefore
\[
\det (J\Sigma_n)(\theta)\ne 0\Leftrightarrow 0\notin\{r,\sin\theta_1,\dots,\sin\theta_{n-2}\}.
\]
So, $\det (J\Sigma_n)(\theta)\ne 0$ almost everywhere.

Mapping $\Pi_n$ from Euclidean coordinates to spherical coordinates is defined by the formulas:
\[
\begin{aligned}r&={\textstyle {\sqrt {{x_{n}}^{2}+{x_{n-1}}^{2}+\cdots +{x_{2}}^{2}+{x_{1}}^{2}}}},\\\theta _{1}&=\operatorname {atan2} \left({\textstyle {\sqrt {{x_{n}}^{2}+{x_{n-1}}^{2}+\cdots +{x_{2}}^{2}}}},x_{1}\right),\\\theta _{2}&=\operatorname {atan2} \left({\textstyle {\sqrt {{x_{n}}^{2}+{x_{n-1}}^{2}+\cdots +{x_{3}}^{2}}}},x_{2}\right),\\&\qquad \vdots \\\theta _{n-2}&=\operatorname {atan2} \left({\textstyle {\sqrt {{x_{n}}^{2}+{x_{n-1}}^{2}}}},x_{n-2}\right),\\\theta _{n-1}&=\operatorname {atan2} \left(x_{n},x_{n-1}\right).\end{aligned}
\]  

At points $0\notin\{r,\sin\theta_1,\dots,\sin\theta_{n-2}\}$ that is $0\notin\{x_1,\dots,x_{n-1}\}$ the Jacobian $(J\Pi_n)(x)$
is full rank as a reverse of a full rank Jacobian.

So, $\det (J\Pi_n)(x)\ne 0$ almost everywhere.

\subsection{Stereographic projection diffeomorphism}

Consider~$\Gamma_n$, the ``standard'' stereographic projection from $S^n$ without the North pole $\zeta=(0,\dots,0,1)$ to $\mathbb{R}^n$:

\begin{align*}
&\Gamma_n(q_1,\dots,q_{n+1}) = \left(\frac{q_1}{1-q_{n+1}},\dots,\frac{q_{n-1}}{1-q_{n+1}}\right);\\
&\Gamma_n^{-1}(x_1,\dots,x_n) = \left(\frac{2x_1}{\lVert x\rVert^2+1},\dots,\frac{2x_{n+1}}{\lVert x\rVert^2+1},\frac{\lVert x\rVert^2-1}{\lVert x\rVert^2+1}\right),
\end{align*}
where $\lVert x\rVert^2 = x_1^2+\dots+x_n^2$.

It is known~\cite{stereo} to be a diffeomorphism.

\subsection{Gnomonic projection}

Gnomonic projection is like stereographic projection, but with the projection hyperplane passing through the center of the hypersphere.

\fxwarning{Below were copied from an AI, check.}

The formulas for projection of a point $p=(p_1,\dots,p_{n+1})$ ($\lVert p\rVert = 1$) of the hypersphere are:

$q=\frac{p}{s\cdot p}$, where $q$ is the projection, $s$ is the point of the sphere. Note that $q_{n+1}=1$. We will exclude this coordinate.

It is evident \fxwarning{Prove.} that gnomonic projection is a diffeomorphism between the hyperplane and ``open-half-hy\-per\-sphe\-re'' ($s\cdot p>0$). Let denote it $S^n_{>}$.

I will denote the Gnomonic projection as~$\Phi_n$.

It is well known that Gnomonic projection maps great circles to lines.

\subsection{The proof}

Let $K$ be a Besicovitch set in $\mathbb{R}^n$.

I will call the \emph{right vicinity} or \emph{one-si\-de vicinity} of a point~$a$ the funcoid~$\supfun{\Delta_{\geq}}\{a\}$.

We will do our geometry in $\mathbb{R}^{n+1}$.

Dimensionality of a funcoid is defined as least dimensionality of a set above it.

Instead of proving the Kakeya conjecture, I will prove a stronger statement:

\begin{claim}
(Funcoidal) join~$L$ of one-si\-de vicinities $v_a$ (laying on great circles) of each point on $S^n_{>}$ hemisphere has dimension~$n$, provided that the set~$K_1$ of directions (measured as their ``projection'' to unit sphere) is $n$-di\-men\-sio\-nal.
\end{claim}

Replace each ``curved'' vicinity $v_a$ by a straight one~$h_a$ (going outside the hypersphere in~$\mathbb{R}^{n+1}$) in the tangential ray determined by~$v_a$. Let $s_a$ be an ``outward'' radial (orthogonal to the hypersphere at point~$a$) one-si\-de vicinity. Then take funcoidal join $H=\bigsqcup_{a\in K_1}(h_a\sqcup s_a)$. Informally, $H$ is a ``swelling'' or ``hairs'' on the hypersphere in $\mathbb{R}^{n+1}$ into both tangential and orthogonal directions.

The straight vicinities $h_a$, projected to the sphere by the radial projection, become exactly original curved vicinities $v_a$ (because they lay on great circles): $\Lambda h_a=v_a$, where $\Lambda$ is the radial projection to the hypersphere.

Thus $L = \Phi_n H$.

If $\Phi_n H$ has dimension~$n$, then the claim is proved.

\begin{lem}
If a set $H$ has dimension~$n+1$, then its radial (toward the center of the hypersphere) projection~$K_1$ to~$S^n$ has dimension~$n$.
\end{lem}

\begin{proof}
Follows from~\cite{189275}\footnote{Or \href{https://grok.com/share/bGVnYWN5_2e0aef8b-4309-420f-815c-a09d76ae97c1}{these AI musings}
that I didn't check for correctness.}.
(We also apply that spherical coordinates are diffeomorphic to Euclidean coordinates, except of a negligible set.)
\end{proof}

It follows, that if we prove that $H$ has dimension~$n+1$, then $K_1$~has dimension~$n$ and our claim is proved.

Let map (name the map $b$\footnote{From \textbf{b}lack hole.}) the sphere together with its swelling to coordinates center by the formula $r\mapsto r-1$ for $r\geq 1$ in spherical coordinates (leaving $\theta'$ the same).
This preserves the dimension because the Jacobian matrix has full rank almost everywhere (hint: The Jacobian is the identity matrix).
Obviously, it is a diffeomorphism of $\setcond{x}{r>1}$ to $\setcond{x}{r\ne 0}$.

\begin{obvious}
$\supfun{b}s_a$ is a vicinity from the center to the direction~$a$.
\end{obvious}

Thus we have enough to prove that the join~$bH$ of one-si\-de vicinities~$w_d$ for a full-me\-a\-su\-re choice~$d$ of directions of the same point (the center of our sphere) is~$n$.

By properties of diffeomorphisms, $\dim bH = \dim H$.

By properties of $n$-fun\-co\-ids, it's evident (because it, with possible exclusion of a finite set, is mapped by~$\Pi_n$ to a funcoidal product of full-measure one-si\-de vicinities on~$\mathbb{R}$) that $\dim bH=n+1$. \fxwarning{Define the product of funcoids.}

\bibliographystyle{amsplain}
\bibliography{kakeya}

\end{document}