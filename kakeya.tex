\documentclass{amsart}
\usepackage{amsmath,amssymb,amsthm}

\newcommand{\funcoids}{\mathsf{FCD}}

\newtheorem{theorem}{Theorem}

\title{Proof Kakeya's conjecture}

\begin{document}

\textbf{Kakeya set conjecture:} Define a \emph{Besicovitch} set in $\mathbb{R}^n$ to be a set which contains a unit line segment in every direction. Is it true that such sets necessarily have Hausdorff dimension and Minkowski dimension equal to~$n$?

\section{Kakeya proof sketch (rough draft)}

Let $K$ be a Besicovitch set in $\mathbb{R}^n$.

I will denote $n$-dimensional spherical coordinates as~$\theta=(r, \theta_1,\dots,\theta_{n-1})$.
I also denote direction $\theta'=(\theta_1,\dots,\theta_{n-1})$.

Here $r\geq 0$; $\theta_i\in[0;\pi]$ for $i=1,\dots,n-2$; $\theta_{n-1}\in[0;2\pi\mathclose[$ .

We can represent $K$ as a union of line segments $K = \bigcup_{\theta'} u_{\theta'}$, where $u_{\theta'}$ is a unit segment of direction~$\theta'$.

It can be reduced to stronger conjecture: for any $\theta'$, \[ K = \bigsqcup^{\funcoids}_{\theta'} (\Delta_{+,\theta'}+s_{\theta'}) \] (where $\Delta_{+,\theta'}$ is the positive vicinity of zero on $\theta'$-di\-rec\-ted line through zero, $s_{\theta'}$ is an arbitrary vector) has dimension~$n$. (We take the dimension of a funcoid the smallest dimension of its containing set.)

We can assume that $\theta'\mapsto s_{\theta'}$ is a dimension preserving function in both directions. Really,
take $s_{\theta'}$ the inverse of the usual mapping from spherical coordinates to $\mathbb{R}^n$. This has
a Jacobian matrix with full rank $n$ almost everywhere, therefore by \cite{gpt-preserve-dim} it preserves dimension.

Then \[ \dim K = \dim\bigsqcup^{\funcoids}_{\theta} (\Delta_{{+,\theta'}}+r) = \dim\bigsqcup^{\funcoids}_{\theta'}(\Delta_{{+,\theta'}}\times[-\pi;\pi]) \]

Map the vicinity of zero of $\mathbb{R}^n$ to angles, preserving dimension. So, we have it equal to
\[ \dim\prod^{\funcoids}\Delta_{+} = n. \]

\bibliographystyle{amsplain}
\bibliography{kakeya}

\end{document}